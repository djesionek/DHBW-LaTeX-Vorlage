%!TEX root = ../ausarbeitung.tex

\chapter{Kapitel 2}
\blindtext


\section{Abschnitt 1}
\begin{figure}		
	\begin{center}
		\begin{tikzpicture}
			\pie[text=legend, radius=3.2]{25.3/{A}, 5.1/{B}, 4.7/{B}, 12.9/{C}, 5.4/{D}, 3.2/{E}, 3.0/{F}, 33.6/{F},2.9/{G},4/{H}}
		\end{tikzpicture}
	\end{center}
	\caption{Zitierte grafik, 2015\cite{Test.2015}}
\end{figure}


\begin{description}
	\item [Erklärung 1] \blindtext
	\item [Erklärung 2] \blindtext
\end{description}
\section{Abschnitt 2}
\begin{figure}		
	\begin{center}
		\begin{tikzpicture}
			\draw [->] (0,0) -- (1,0);
			\draw [black] (1,-0.75) rectangle (3,0.75); % W=2, H=1.5

			\draw [->] (3,0) -- (4,0);
			\draw [black] (4,-0.75) rectangle (6,0.75);

			\draw [->] (6,0) -- (7,0);
			\draw [black] (7,-0.75) rectangle (9,0.75);

			\draw [->] (9,0) -- (10,0);
			\draw [black] (10,-0.75) rectangle (12,0.75);

			\draw [->] (11, -1.5) -- (11, -0.75);
			\draw [black, rounded corners] (10,-2.5) rectangle (12, -1.5);

			\draw [->] (12,0) -- (13,0);

			% Line Labels
			\node [above] at (-0.25,0) {A};
			\node [above] at (3.5,1) {B};
			\node [above, align=center] at (6.5,1) {C};
			\node [above, align=center] at (9.5,1) {D};
			\node [above] at (13,0) {E};

			% Rectangle Labels
			\node [above, align=center] at (2,-0.75) {A};
			\node [above, align=center] at (5,-0.75) {B};
			\node [above, align=center] at (8,-0.75) {C};
			\node [above, align=center] at (11,-0.75) {D};
			\node [above, align=center] at (11,-2.25) {E};
		\end{tikzpicture}
	\end{center}
	\caption{Random Ablauf}
\end{figure}
